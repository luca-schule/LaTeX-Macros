\documentclass[10pt,a4paper, ngerman]{beamer}
				
\include{beamer-with-code}

\usepackage[style=authortitle,backend=biber, natbib=true, hyperref=true]{biblatex}
\addbibresource{mybib.bib}
\makeindex

\AtBeginSection{\frame{\frametitle{Gliederung}\tableofcontents[currentsection]}}

\setbeamercovered{transparent}

\title{Makros in \LaTeX}
%\subtitle{subtitle}
\date{\today}
\institute[HBBK]{Hans-Böckler-Berufskolleg}
\setlength{\itemsep}{10pt}
\begin{document}
\begin{frame}
\titlepage
\end{frame}

\section{Befehle in \LaTeX}

\begin{frame}[fragile]
\frametitle{Befehle in \LaTeX}
\begin{lstlisting}[caption = Befehle]
In einem \LaTeX \ Dokument gibt es verschiedene \textbf{Befehle}, \\
die Text verschieden \textit{darstellen} können.
\end{lstlisting} 
\begin{hlbox}
In einem \LaTeX \ Dokument gibt es verschiedene \textbf{Befehle}, \\
die Text verschieden \textit{darstellen} können.
\end{hlbox}
\end{frame}


\section{Makros}
\begin{frame}{Makros}
\begin{itemize} 
\item Makros sind selbst-definierte Befehle
\pause
\item Definition neuer, oder Redifinition alter Befehle
\pause
\item Standardmäßig in \LaTeX \ integriert
\end{itemize}
\end{frame}

\subsection{Wichtige Befehle}

\begin{frame}[fragile]{Wichtige Befehle}
\begin{tabular}{l|p{0.7\textwidth}}
\textbf{Befehl} & \textbf{Beschreibung} \\ \hline
\lstinline|\newcommand| & Erstellt neuen Befehl. Fehler, wenn der Befehl schon existiert \\ \pause
\lstinline|\renewcommand| & Überschreibt einen bestehenden Befehl. Fehler, wenn der Befehl noch nicht existiert \\ \pause
\lstinline|\providecommand| & Erstellt neuen Befehl. Wird ignoriert, wenn Befehl existiert \\ \pause
\lstinline|\newenvironment| & Erstellt neue Umgebung. Fehler, wenn diese bereits existiert  \\ \pause
\lstinline|\renewenvironment|  & Überschreibt eine Umgebung. Fehler, wenn diese noch nicht existiert \\
\end{tabular}
\end{frame}

\section{Befehle definieren}
\subsection{Einfache Makros}

\begin{frame}[fragile]{Befehle definieren}{Einfache Makros}
\newcommand{\R}{\( \mathbb{R} \)}
\begin{lstlisting}[caption = newcommand]
\newcommand{\R}{\(\mathbb{R}\)}
Die Zahl n ist Teil der Menge der reellen Zahlen \R 
\end{lstlisting}
\begin{hlbox}
Die Zahl n ist Teil der Menge der reellen Zahlen \R 
\end{hlbox}
\end{frame}

\begin{frame}[fragile]{Befehle definieren}{Einfache Makros}
\begin{lstlisting}
\newcommand{\R}{\(\mathbb{R}\)}
\end{lstlisting}
\begin{itemize}
\item \lstinline|\R|: Name des neuen Befehls
\item \lstinline|\(\mathbb{R}\)|: Abfolge an ausgeführten Befehlen
\end{itemize}
\end{frame}

\subsection{Makros mit Parametern}

\begin{frame}[fragile]{Befehle definieren}{Makros mit Parametern}
\newcommand{\mz}[1]{\(\mathbb{#1}\)}
\begin{lstlisting}[caption = Mengenzeichen]
\newcommand{\mz}[1]{\(\mathbb{#1}\)}
Wenn die Zahl n Teil der Menge der ganzen Zahlen \mz{Z} ist, ist sie auch Teil der Menge der reellen Zahlen \mz{R}
\end{lstlisting}
\begin{hlbox}
Wenn die Zahl n Teil der Menge der ganzen Zahlen \mz{Z} ist, ist sie auch Teil der Menge der reellen Zahlen \mz{R}
\end{hlbox}
\end{frame}

\begin{frame}[fragile]{Befehle definieren}{Makros mit Parametern}
\begin{lstlisting}
\newcommand{\mz}[1]{\(\mathbb{#1}\)}
\end{lstlisting}
\begin{itemize}
\item \lstinline|\mz|  ist der Name des neuen Befehls.
\item \lstinline|[1]|  ist die Anzahl der akzeptierten Parameter
\item \lstinline|\(\mathbb{#1}\)|  ist der ausgeführte Befehl. 
\end{itemize}
\end{frame}
\newcommand{\bp}[3][2]{\((#2 + #3) ^ #1\)}
\subsection{Makros mit optinalen Parametern}
\begin{frame}[fragile]{Befehle definieren}{Makros mit optinalen Parametern}
\begin{lstlisting}[caption = binomisch Plus]
\newcommand{\bp}[3][2]{\((#2 + #3) ^ #1\)}
\bp{x}{y}
\end{lstlisting}
\begin{hlbox}
\bp{x}{y}
\end{hlbox}
\end{frame}

\begin{frame}[fragile]{Befehle definieren}{Makros mit optionalen Parametern}
\begin{lstlisting}
\newcommand{\bp}[3][2]{\((#2 + #3) ^ #1\)}
\end{lstlisting}
\begin{itemize}
\item \lstinline|\bp| den Namen des neuen Befehls an. 
\item \lstinline|[3]| ist die Anzahl der Parameter. 
\item \lstinline|[2]| ist der Standardwert des ersten Parameters
\item \lstinline|\((#2 + #3) ^ #1\)| ist der ausgeführte Befehl.
\end{itemize}
\end{frame}

\begin{frame}[fragile]{Befehle definieren}{Makros mit optionalen Parametern}
\begin{lstlisting}[caption = binomisch Plus hoch 4]
\bp[4]{32}{48}
\end{lstlisting}
\begin{hlbox}
\bp[4]{32}{48}
\end{hlbox}
\begin{itemize}
\item \lstinline|[4]| Standardparameter überschreibender Wert
\end{itemize}
\end{frame}

\section{Bestehende Befehle überschreiben}
\begin{frame}[fragile]{Bestehende Befehle überschreiben}
\begin{itemize}
\item Befehl: \lstinline|\renewcommand|
\pause
\item Vom System verwendete Befehle können überschrieben werden
\item Fehler wenn Befehl noch nicht definiert
\end{itemize}
\end{frame}

\begin{frame}[fragile]{Bestehende Befehle überschreiben}
\begin{lstlisting}[caption = kleine Fußnoten]
% Setze die größe der Fußnoten auf klein
\renewcommand{\footnotesize}{\small}
\end{lstlisting}
\pause
Der Syntax ist dem von \lstinline|\newcommand| sehr ähnlich:
\begin{itemize}
\item \lstinline|\footnotesize| der Name des Befehls, der erneuert werden soll. \pause
\item \lstinline|\small| ist der Befehl, durch den \lstinline|\footnotesize| ersetzt wird. 
\end{itemize}
\end{frame}

\begin{frame}[fragile]{Bestehende Befehle überschreiben}
\begin{lstlisting}[caption = groß und klein]
\large Groß \\
\small Klein
\end{lstlisting}
\begin{hlbox}
\large Groß \\
\small Klein
\end{hlbox}
\end{frame}

\section{Umgebungen definieren}
\begin{frame}[fragile]{Umgebungen definieren}
\begin{itemize}
\pftn{{L}a{T}e{X}-{W}örterbuch: {U}mgebung --- {W}ikibooks{,} {D}ie freie {B}ibliothek}
\item Wirken auf begrenzten Textbereich \pause
\item Begrenzt durch \lstinline|\begin{...}| und \lstinline|\end{...}| \pause
\item Änderungen werden am Ende der Umgebung aufgehoben  
\end{itemize}
\end{frame}

\begin{frame}[fragile]{Umgebungen definieren}
\begin{lstlisting}[caption = zentrierter Text]
\begin{center}
Dieser Text ist zentriert
\end{center}
\end{lstlisting}
\pause
\begin{hlbox}
\begin{center}
Dieser Text ist zentriert
\end{center}
\end{hlbox}
\pause
\begin{itemize}
\item \lstinline|\begin{center}| ist der Anfang des \texttt{center}-Blocks.  \pause
\item \enquote{Dieser Text ist zentriert} ist der zentrierte Text innerhalb des \texttt{center}-Blocks. \pause
\item \lstinline|\end{center}| ist das Ende des \texttt{center}-Blocks. 
\end{itemize}
\end{frame}

\begin{frame}[fragile]{Umgebungen definieren}
\begin{lstlisting}[caption = Highlightbox]
\newenvironment{hlbox}
{\begin{tcolorbox}[enhanced,colback=white,colframe=white,sharpish corners,fuzzy halo=0.5mm with lightgray]}
{\end{tcolorbox}}
\end{lstlisting} \pause
\begin{itemize}
\item \lstinline|\newenvironment{hlbox}|: Name der neuen Umgebung \pause
\item \lstinline|{\begin{tcolorbox}|: Erster Teil der Umgebung \pause
\item \lstinline|[enhanced,colb...| Optionen für \texttt{tcolorbox} \pause
\item \lstinline|{\end{tcolorbox}}|: Letzter Teil der Umgebung
\end{itemize}
\end{frame}
\renewenvironment{bmatrix}
{ \begin{center} \begin{em} }
{ \end{em} \end{center} }
\section{Bestehende Umgebungen überschreiben}
\begin{frame}[fragile]{Bestehende Umgebungen überschreiben}
\begin{lstlisting}[caption=Matrix]
\renewenvironment{bmatrix}
{ \begin{center} \begin{em} }
{ \end{em} \end{center} }

\begin{bmatrix}
zentrierter Text
\end{bmatrix}
\end{lstlisting} \pause
\begin{hlbox}
\begin{bmatrix}
zentrierter Text
\end{bmatrix}
\end{hlbox}
\end{frame}

\begin{frame}[fragile]{Bestehende Umgebungen überschreiben}
\begin{lstlisting}[caption=Matrix]
\renewenvironment{bmatrix}
{ \begin{center} \begin{em} }
{ \end{em} \end{center} }
\end{lstlisting} \pause
\begin{itemize}
\item \lstinline|bmatrix|: Name der Umgebung \pause
\item \lstinline|{ \begin{center} \begin{em} }|: Teil vor dem eingesetzten Text \pause
\item \lstinline|{ \end{em} \end{center} }|: Teil, der 2. beendet 
\end{itemize}
\end{frame}

\end{document}