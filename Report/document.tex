\documentclass[ngerman,12pt,titlepage]{scrartcl}

\usepackage[ngerman]{babel}

\usepackage[T1]{fontenc}
\usepackage[utf8]{inputenc}
\usepackage{varioref}
\usepackage{hyperref}
\usepackage{cleveref}
\usepackage{makeidx}
\usepackage{csquotes}
\usepackage{listings}
\usepackage[style=authoryear,backend=biber, natbib=true, hyperref=true]{biblatex}
\addbibresource{mybib.bib}
\makeindex
\usepackage{graphicx}
\usepackage{listings}
\usepackage{color}
\usepackage{xcolor}
\usepackage[most]{tcolorbox}
 
\definecolor{codegreen}{rgb}{0,0.6,0}
\definecolor{codegray}{rgb}{0.5,0.5,0.5}
\definecolor{codepurple}{rgb}{0.58,0,0.82}
\definecolor{backcolour}{rgb}{0.95,0.95,0.92}
 
\lstdefinestyle{mystyle}{
    backgroundcolor=\color{backcolour},   
    commentstyle=\color{codegreen},
    keywordstyle=\color{magenta},
    numberstyle=\tiny\color{codegray},
    stringstyle=\color{codepurple},
    basicstyle=\footnotesize,
    breakatwhitespace=false,         
    breaklines=true,                 
    captionpos=b,                    
    keepspaces=true,                 
    numbers=left,                    
    numbersep=5pt,                  
    showspaces=false,                
    showstringspaces=false,
    showtabs=false,                  
    tabsize=2
}
 
\lstset{style=mystyle}

\title{Makros in \LaTeX}
\author{Luca Kiebel}
\date{\today}

% Im Dokument genutzte Makros
\newenvironment{hlbox}{\begin{tcolorbox}[enhanced,colback=white,colframe=white,sharpish corners,fuzzy halo=0.5mm with lightgray]}{\end{tcolorbox}}

\begin{document}
\maketitle
\newpage
	\tableofcontents
\newpage

\section{Einleitung}
Die meisten \LaTeX Befehle sind einfache Worte mit einem Backslash davor:

\begin{lstlisting}
In einem \LaTeX \ Dokument gibt es verschiedene \textbf{Befehle}{,} \\
die Text verschieden \textit{darstellen}.
\end{lstlisting} 
\begin{hlbox}
In einem \LaTeX \ Dokument gibt es verschiedene \textbf{Befehle}{,} \\
die Text verschieden \textit{darstellen}.
\end{hlbox}
Makros sind selbstdefinierte Befehle, die entweder alte Kommandos überschreiben oder neue definieren.
Sie sind standardmäßig in \LaTeX \ integriert und müssen nicht als Paket eingefügt werden. 

\section{Befehle definieren}

\LaTeX \ kommt mit einer großen Anzahl von Befehlen für viele Aufgaben, dennoch ist es manchmal notwendig, einige spezielle Befehle zu definieren, um wiederholte oder komplexe Formatierungen zu vereinfachen und zu verkürzen. Hier kommen Makros ins Spiel.

\subsection{Einfache Makros}
Neue Befehle werden durch die {\textbackslash}newcommand-Anweisung definiert. Will man zum Beispiel alle Reellen Zahlen ausgeben{,} so kann der Befehl \lstinline|{}| einfach durch das Makro \lstinline|\R| ersetzt werden:
\begin{lstlisting}
\newcommand{\R}{\mathbb{R}}
\end{lstlisting}

Dieser neue Befehl kann dann im \LaTeX \ Code an jeder Stelle genutzt werden{,} $  $
\newpage
\printbibliography

\end{document}
